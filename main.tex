\documentclass{amsart}

%\usepackage{etoolbox}
%\makeatletter
%\let\ams@starttoc\@starttoc
%\makeatother
%\makeatletter
%\let\@starttoc\ams@starttoc
%\patchcmd{\@starttoc}{\makeatletter}{\makeatletter\parskip\z@}{}{}
%\makeatother

%\usepackage[parfill]{parskip}

\usepackage[colorlinks=true,linkcolor=blue,citecolor=blue,urlcolor=blue]{hyperref}
\usepackage{bookmark}
\usepackage{amsthm,thmtools,amssymb,amsmath,amscd}

\usepackage[bibstyle=alphabetic,citestyle=alphabetic,backend=bibtex]{biblatex}
\bibliography{Bibliography}

\usepackage{fancyhdr}
\usepackage{esint}

\usepackage{enumerate}

\usepackage{pictexwd,dcpic}

\usepackage{graphicx}

\swapnumbers
\declaretheorem[name=Theorem,numberwithin=section]{thm}
\declaretheorem[name=Remark,style=remark,sibling=thm]{rem}
\declaretheorem[name=Lemma,sibling=thm]{lemma}
\declaretheorem[name=Proposition,sibling=thm]{prop}
\declaretheorem[name=Definition,style=definition,sibling=thm]{defn}
\declaretheorem[name=Corollary,sibling=thm]{cor}
\declaretheorem[name=Assumption,style=remark,sibling=thm]{ass}
\declaretheorem[name=Example,style=remark,sibling=thm]{example}


\numberwithin{equation}{section}

\usepackage{cleveref}
\crefname{lemma}{Lemma}{Lemmata}
\crefname{prop}{Proposition}{Propositions}
\crefname{thm}{Theorem}{Theorems}
\crefname{cor}{Corollary}{Corollaries}
\crefname{defn}{Definition}{Definitions}
\crefname{example}{Example}{Examples}
\crefname{rem}{Remark}{Remarks}
\crefname{ass}{Assumption}{Assumptions}
\crefname{not}{Notation}{Notation}

%Symbols
\renewcommand{\~}{\tilde}
\renewcommand{\-}{\bar}
\newcommand{\bs}{\backslash}
\newcommand{\cn}{\colon}
\newcommand{\sub}{\subset}

\newcommand{\N}{\mathbb{N}}
\newcommand{\R}{\mathbb{R}}
\newcommand{\Z}{\mathbb{Z}}
\renewcommand{\S}{\mathbb{S}}
\renewcommand{\H}{\mathbb{H}}
\newcommand{\C}{\mathbb{C}}
\newcommand{\K}{\mathbb{K}}
\newcommand{\Di}{\mathbb{D}}
\newcommand{\B}{\mathbb{B}}
\newcommand{\8}{\infty}

%Greek letters
\renewcommand{\a}{\alpha}
\renewcommand{\b}{\beta}
\newcommand{\g}{\gamma}
\renewcommand{\d}{\delta}
\newcommand{\e}{\epsilon}
\renewcommand{\k}{\kappa}
\renewcommand{\l}{\lambda}
\renewcommand{\o}{\omega}
\renewcommand{\t}{\theta}
\newcommand{\s}{\sigma}
\newcommand{\p}{\varphi}
\newcommand{\z}{\zeta}
\newcommand{\vt}{\vartheta}
\renewcommand{\O}{\Omega}
\newcommand{\D}{\Delta}
\newcommand{\G}{\Gamma}
\newcommand{\T}{\Theta}
\renewcommand{\L}{\Lambda}

%Mathcal Letters
\newcommand{\cL}{\mathcal{L}}
\newcommand{\cT}{\mathcal{T}}
\newcommand{\cA}{\mathcal{A}}
\newcommand{\cW}{\mathcal{W}}

%Mathematical operators
\newcommand{\INT}{\int_{\O}}
\newcommand{\DINT}{\int_{\d\O}}
\newcommand{\Int}{\int_{-\infty}^{\infty}}
\newcommand{\del}{\partial}

\newcommand{\inpr}[2]{\left\langle #1,#2 \right\rangle}
\newcommand{\fr}[2]{\frac{#1}{#2}}
\newcommand{\x}{\times}
\DeclareMathOperator{\Tr}{Tr}

\DeclareMathOperator{\dive}{div}
\DeclareMathOperator{\id}{id}
\DeclareMathOperator{\pr}{pr}
\DeclareMathOperator{\Diff}{Diff}
\DeclareMathOperator{\supp}{supp}
\DeclareMathOperator{\graph}{graph}
\DeclareMathOperator{\osc}{osc}
\DeclareMathOperator{\const}{const}
\DeclareMathOperator{\dist}{dist}
\DeclareMathOperator{\loc}{loc}
\DeclareMathOperator{\grad}{grad}
\DeclareMathOperator{\ric}{Ric}
\DeclareMathOperator{\Rm}{Rm}
\DeclareMathOperator{\weingarten}{\mathcal{W}}
\DeclareMathOperator{\inj}{inj}

%Environments
\newcommand{\Theo}[3]{\begin{#1}\label{#2} #3 \end{#1}}
\newcommand{\pf}[1]{\begin{proof} #1 \end{proof}}
\newcommand{\eq}[1]{\begin{equation}\begin{alignedat}{2} #1 \end{alignedat}\end{equation}}
\newcommand{\IntEq}[4]{#1&#2#3	 &\quad &\text{in}~#4,}
\newcommand{\BEq}[4]{#1&#2#3	 &\quad &\text{on}~#4}
\newcommand{\br}[1]{\left(#1\right)}



%Logical symbols
\newcommand{\Ra}{\Rightarrow}
\newcommand{\ra}{\rightarrow}
\newcommand{\hra}{\hookrightarrow}
\newcommand{\mt}{\mapsto}

% Aleksandrov Reflection Macros
\DeclareMathOperator{\reflectionvector}{V}
\DeclareMathOperator{\reflectionangle}{\delta}
\newcommand{\reflectionplane}[1][\reflectionvector]{\ensuremath{P_{#1}}}
\newcommand{\reflectionmap}[1][\reflectionvector]{\ensuremath{R_{#1}}}
\newcommand{\reflectionset}[2][\reflectionvector]{\ensuremath{{#2}_{#1}}}
\newcommand{\reflectionhalfspace}[1][\reflectionvector]{\ensuremath{\reflectionset[{#1}]{H}}}
\DeclareMathOperator{\vertvec}{e}
\DeclareMathOperator{\origin}{O}
\DeclareMathOperator{\radialprojection}{\pi}
\DeclareMathOperator{\height}{h}
\DeclareMathOperator{\equator}{E}
\newcommand{\ip}[2]{\ensuremath{\langle{#1},{#2}\rangle}}
\DeclareMathOperator{\intersect}{\cap}
\DeclareMathOperator{\union}{\cup}
\DeclareMathOperator{\nor}{\nu}
\DeclareMathOperator{\basepoint}{p_0}
\DeclareMathOperator{\radialdistance}{r}

%Fonts
\newcommand{\mc}{\mathcal}
\renewcommand{\it}{\textit}
\newcommand{\mrm}{\mathrm}

%Spacing
\newcommand{\hp}{\hphantom}


%\parindent 0 pt

\protected\def\ignorethis#1\endignorethis{}
\let\endignorethis\relax
\def\TOCstop{\addtocontents{toc}{\ignorethis}}
\def\TOCstart{\addtocontents{toc}{\endignorethis}}


\begin{document}

\title{Sharp Harnack Inequality}

\curraddr{}
\email{}
\date{\today}

\dedicatory{}
\subjclass[2010]{}
\keywords{}

\maketitle

Let $u$ be a solution of the heat equation on a Riemannian manifold $(M, g)$,
\[
\partial_t u = \Delta_g u.
\]
For simplicity, assume $M$ is compact and $\ric \geq 0$. It's possible to get away with less: $\ric \geq - K_0$ or possibly even something like $\ric \geq -K_0 \rho(x)^2$ where $\rho(x) = d(x, x_0)$ is the distance from a fixed point and $K_0 \geq 0$.

The Harnack inequality is of the form
\[
u(x_1, t_1) \leq C u(x_2, t_1)
\]
for $x_1, x_2 \in K$ a compact set, $t_2 > t_1 > 0$ and
\[
C = C(t_1, t_2, \operatorname{diam} (K), g).
\]

To find the best $C$ we note that an equivalent formulation is
\[
\ln u(x_2, t_2) - \ln u(x_1, t_1) \geq - \ln C.
\]
For convenience, let $f = \ln u$. Then equivalently we want
\[
f(x_2, t_2) - f(x_1, t_1) = \int_{t_1}^{t_2} \tfrac{d}{dt} [f(\gamma(t), t)] dt = \int_{t_1}^{t_2} \partial_t f + df \cdot \gamma' dt \geq - \ln C
\]
where $\gamma$ is any curve with $\gamma(t_1) = x_1$ and $\gamma(t_2) = x_2$. The inequality,
\[
df \cdot \gamma = g(\nabla f, \gamma') \geq -\abs{\nabla f}^2 - \frac{1}{4} \abs{\gamma'}^2
\]
allows us to separate the $\nabla f$ part and the $\gamma'$ part with equality attained precisely when $\gamma' = -2\nabla f$. This suggests the optimum $C$ will be obtained from a point-wise lower bound for
\[
Q = \partial_t f - \abs{\nabla f}^2 = \Delta Q.
\]

Then using the Bochner formula and Cauchy Schwartz applied to $\abs{\nabla^2 f}^2$ we compute,
\[
\partial_t Q \geq \Delta Q + g(\nabla f, \nabla Q) + \frac{2}{n} Q^2
\]
and the maximum principle implies
\[
Q \geq -\frac{n}{2} \frac{1}{t}.
\]
That is,
\[
\partial_t f - \abs{\nabla f}^2 \geq -\frac{n}{2} \frac{1}{t}
\]
which gives
\[
\partial_t f + d \cdot \gamma' \geq - \frac{1}{4} \abs{\gamma'}^2 - \frac{n}{2} \frac{1}{t}
\]

Integrating we obtain
\[
\ln \left(\frac{u(x_2, t_2)}{u(x_1, t_1)}\right) = f(x_2, t_2) - f(x_1, t_1) \geq - \frac{n}{2} \int_{t_1}^{t_2} \frac{1}{t} - \frac{1}{4} \int_{t_1}^{t_2} \abs{\gamma'}^2 dt = + \ln\left[\left(\frac{t_2}{t_1}\right)^{-\tfrac{n}{2}}\right] - \frac{1}{4} \int_{t_1}^{t_2} \abs{\gamma'}^2 dt.
\]
Since this is true for every $\gamma$ we can take the sup of the right hand side to obtain
\[
\ln \left(\frac{u(x_2, t_2)}{u(x_1, t_1)}\right) \geq  \ln\left[\left(\frac{t_2}{t_1}\right)^{-\tfrac{n}{2}}\right] - \frac{d(x_1, x_2)^2}{4(t_2 - t_1)}
\]

Taking exponentials yields,
\[
\frac{u(x_2, t_2)}{u(x_1, t_1)} \geq \left(\frac{t_2}{t_1}\right)^{-\tfrac{n}{2}} \exp\left(- \frac{d(x_1, x_2)^2}{4(t_2 - t_1)}\right).
\]
Thus
\[
u(x_1, t_1) \leq \left[\left(\frac{t_2}{t_1}\right)^{\tfrac{n}{2}} \exp\left(\frac{d(x_1, x_2)^2}{4(t_2 - t_1)}\right)\right] u(x_2, t_2)
\]
and we take,
\[
C = \left(\frac{t_2}{t_1}\right)^{\tfrac{n}{2}} \exp\left(\frac{d(x_1, x_2)^2}{4(t_2 - t_1)}\right)
\]

Now consider the fundamental solution in Euclidean space,
\[
u(x, t) = \frac{1}{(4\pi t)^{n/2}} \exp\left(-\frac{\abs{x}^2}{4t}\right)
\]
centered on $(x_0, t_0) = (0, 0)$. This is a similarity solution:
\[
u(x, t) = \lambda(t) u_0(\varphi_t(x))
\]
where $\lambda(t) = (4\pi t)^{-n/2}$, $\varphi_t(x) = \tfrac{1}{2\sqrt{t}} x$ and $u_0(x) = \exp(-\abs{x}^2)$.

By the Harnack inequality, for any $x_1, x_2$, $t_2 > t_1$ we have
\[
\frac{u(x_1, t_1)}{u(x_2, t_2)} = \left(\frac{t_2}{t_1}\right)^{n/2} \exp\left(-\frac{\abs{x_1}^2}{4t_1} + \frac{\abs{x_2}^2}{4t_2}\right) \leq \left(\frac{t_2}{t_1}\right)^{n/2} \exp\left(\frac{\abs{x_1 - x_2}^2}{4(t_2 - t_1)}\right)
\]
In particular, for $x_1 = 0$, this is equivalent to
\[
\exp\left(\frac{\abs{x_2}^2}{4t_2}\right) \leq \exp\left(\frac{\abs{x_2}^2}{4(t_2 - t_1)}\right)
\]
which is certainly true for $t_2 > t_1 > 0$. Now sending $t_1 \to 0$ we get equality so we cannot hope to improve $C$ in the Harnack inequality.

Note also that the $Q$ for the fundamental solution should satisfy,
\[
\partial_t Q = \Delta Q + g(\nabla f, \nabla Q) + \frac{2}{n} Q^2
\]
and
\[
\liminf_{t\to 0} \inf_{x \in \R^n} Q(x, t) = - \infty.
\]
which would complete the picture of "fundamental solution achieves equality".
\end{document}
