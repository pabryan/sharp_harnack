\documentclass{amsart}

\input{StandardPaper2.tex}

\begin{document}

\title{Sharp Harnack Inequality}

\curraddr{}
\email{}
\date{\today}

\dedicatory{}
\subjclass[2010]{}
\keywords{}

\maketitle

Let $u$ be a solution of the heat equation on a Riemannian manifold $(M, g)$,
\[
\partial_t u = \Delta_g u.
\]
For simplicity, assume $M$ is compact and $\ric \geq 0$. It's possible to get away with less: $\ric \geq - K_0$ or possibly even something like $\ric \geq -K_0 \rho(x)^2$ where $\rho(x) = d(x, x_0)$ is the distance from a fixed point and $K_0 \geq 0$.

The Harnack inequality is of the form
\[
u(x_1, t_1) \leq C u(x_2, t_1)
\]
for $x_1, x_2 \in K$ a compact set, $t_2 > t_1 > 0$ and
\[
C = C(t_1, t_2, \operatorname{diam} (K), g).
\]

To find the best $C$ we note that an equivalent formulation is
\[
\ln u(x_2, t_2) - \ln u(x_1, t_1) \geq - \ln C.
\]
For convenience, let $f = \ln u$. Then equivalently we want
\[
f(x_2, t_2) - f(x_1, t_1) = \int_{t_1}^{t_2} \tfrac{d}{dt} [f(\gamma(t), t)] dt = \int_{t_1}^{t_2} \partial_t f + df \cdot \gamma' dt \geq - \ln C
\]
where $\gamma$ is any curve with $\gamma(t_1) = x_1$ and $\gamma(t_2) = x_2$. The inequality,
\[
df \cdot \gamma = g(\nabla f, \gamma') \geq -\abs{\nabla f}^2 - \frac{1}{4} \abs{\gamma'}^2
\]
allows us to separate the $\nabla f$ part and the $\gamma'$ part with equality attained precisely when $\gamma' = -2\nabla f$. This suggests the optimum $C$ will be obtained from a point-wise lower bound for
\[
Q = \partial_t f - \abs{\nabla f}^2 = \Delta Q.
\]

Then using the Bochner formula and Cauchy Schwartz applied to $\abs{\nabla^2 f}^2$ we compute,
\[
\partial_t Q \geq \Delta Q + g(\nabla f, \nabla Q) + \frac{2}{n} Q^2
\]
and the maximum principle implies
\[
Q \geq -\frac{n}{2} \frac{1}{t}.
\]
That is,
\[
\partial_t f - \abs{\nabla f}^2 \geq -\frac{n}{2} \frac{1}{t}
\]
which gives
\[
\partial_t f + d \cdot \gamma' \geq - \frac{1}{4} \abs{\gamma'}^2 - \frac{n}{2} \frac{1}{t}
\]

Integrating we obtain
\[
\ln \left(\frac{u(x_2, t_2)}{u(x_1, t_1)}\right) = f(x_2, t_2) - f(x_1, t_1) \geq - \frac{n}{2} \int_{t_1}^{t_2} \frac{1}{t} - \frac{1}{4} \int_{t_1}^{t_2} \abs{\gamma'}^2 dt = + \ln\left[\left(\frac{t_2}{t_1}\right)^{-\tfrac{n}{2}}\right] - \frac{1}{4} \int_{t_1}^{t_2} \abs{\gamma'}^2 dt.
\]
Since this is true for every $\gamma$ we can take the sup of the right hand side to obtain
\[
\ln \left(\frac{u(x_2, t_2)}{u(x_1, t_1)}\right) \geq  \ln\left[\left(\frac{t_2}{t_1}\right)^{-\tfrac{n}{2}}\right] - \frac{d(x_1, x_2)^2}{4(t_2 - t_1)}
\]

Taking exponentials yields,
\[
\frac{u(x_2, t_2)}{u(x_1, t_1)} \geq \left(\frac{t_2}{t_1}\right)^{-\tfrac{n}{2}} \exp\left(- \frac{d(x_1, x_2)^2}{4(t_2 - t_1)}\right).
\]
Thus
\[
u(x_1, t_1) \leq \left[\left(\frac{t_2}{t_1}\right)^{\tfrac{n}{2}} \exp\left(\frac{d(x_1, x_2)^2}{4(t_2 - t_1)}\right)\right] u(x_2, t_2)
\]
and we take,
\[
C = \left(\frac{t_2}{t_1}\right)^{\tfrac{n}{2}} \exp\left(\frac{d(x_1, x_2)^2}{4(t_2 - t_1)}\right)
\]

Now consider the fundamental solution in Euclidean space,
\[
u(x, t) = \frac{1}{(4\pi t)^{n/2}} \exp\left(-\frac{\abs{x}^2}{4t}\right)
\]
centered on $(x_0, t_0) = (0, 0)$. This is a similarity solution:
\[
u(x, t) = \lambda(t) u_0(\varphi_t(x))
\]
where $\lambda(t) = (4\pi t)^{-n/2}$, $\varphi_t(x) = \tfrac{1}{2\sqrt{t}} x$ and $u_0(x) = \exp(-\abs{x}^2)$.

By the Harnack inequality, for any $x_1, x_2$, $t_2 > t_1$ we have
\[
\frac{u(x_1, t_1)}{u(x_2, t_2)} = \left(\frac{t_2}{t_1}\right)^{n/2} \exp\left(-\frac{\abs{x_1}^2}{4t_1} + \frac{\abs{x_2}^2}{4t_2}\right) \leq \left(\frac{t_2}{t_1}\right)^{n/2} \exp\left(\frac{\abs{x_1 - x_2}^2}{4(t_2 - t_1)}\right)
\]
In particular, for $x_1 = 0$, this is equivalent to
\[
\exp\left(\frac{\abs{x_2}^2}{4t_2}\right) \leq \exp\left(\frac{\abs{x_2}^2}{4(t_2 - t_1)}\right)
\]
which is certainly true for $t_2 > t_1 > 0$. Now sending $t_1 \to 0$ we get equality so we cannot hope to improve $C$ in the Harnack inequality.

Note also that the $Q$ for the fundamental solution should satisfy,
\[
\partial_t Q = \Delta Q + g(\nabla f, \nabla Q) + \frac{2}{n} Q^2
\]
and
\[
\liminf_{t\to 0} \inf_{x \in \R^n} Q(x, t) = - \infty.
\]
which would complete the picture of "fundamental solution achieves equality".
\end{document}
